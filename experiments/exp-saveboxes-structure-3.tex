% !Mode:: "TeX:DE:UTF-8:Main"
\DocumentMetadata{uncompress,testphase=phase-III,xmp=false}
\documentclass{article}
\newsavebox\testbox
\ExplSyntaxOn
\newcommand\teststructure[1]{\tagstructbegin{tag=Span}\tagmcbegin{}#1\tagmcend\tagstructend}
\newcommand\interruptmc[1]{\tag_mc_end_push:#1\tag_mc_begin_pop:n{}}
\newcommand\stashstructure[2]{\tagstructbegin{tag=NonStruct,label=#1,stash}#2\tagstructend}

\ExplSyntaxOn\makeatletter
% this should go when the format has it ...
\def\newsavebox#1{\@ifdefinable{#1}{\newbox#1 \tl_new:c{l_tag_box_\int_use:N#1_tl}}}

\AddToHookWithArguments{cmd/sbox/before}
  {
   \tl_set:cx {l_tag_box_\int_use:N#1_tl}{\int_eval:n{\tag_get:n{struct_counter}+\tag_get:n{mc_counter}}}
  }
\AddToHookWithArguments {cmd/sbox/after}
  {
    \tl_set:cx {l_tag_box_\int_use:N#1_tl}
      {\int_eval:n{\tag_get:n{struct_counter}+\tag_get:n{mc_counter}-\tl_use:c{l_tag_box_\int_use:N#1_tl}}}    
  }
\makeatother  
\ExplSyntaxOff


\newsavebox\mybox
\begin{document}

\section{Saving and using boxes}

\TeX{} allows to store material in boxes and to use these box once or multiple times in other places.
This poses some challenges to tagging. The listings in the following examples uses low-level \TeX{} box
commands to avoid that changes in the \LaTeX{} commands that improve tagging interfere in case you want to 
test this.

\subsection{Boxes without tagging commands}


If no tagging commands were used (or if they were inactive) when the box was stored then
there is no problem to use this box with pdf\LaTeX{}/generic mode in various places. So

\begin{verbatim}
\newbox\mybox
The\setbox\mybox\hbox{yellow} duck

The \box\mybox{} sun
\end{verbatim}

will produce (assuming para tagging is activated) the paragraph structures \enquote{The duck} and \enquote{The yellow sun}.


With lua\LaTeX{}/lua mode this is different: The nodes in the box will have the mc-attribute value attached which were
active when the box was saved and this value is recorded as kid of the first paragraph. So when the lua code later wanders through the box to find all kids of structure it will also find the content of the \cs{usebox}. This means with lua\LaTeX{} we get the two paragraph structures \enquote{The duck yellow} and \enquote{The sun}. 

The solution here is to reset the attributes before using the box:

\begin{verbatim}
\ExplSyntaxOn
\let\tagmcresetbox\tag_mc_reset_box:N
\ExplSyntaxOff

The\setbox\mybox\hbox{yellow} duck

The \tagmcresetbox\mybox\box\mybox{} sun
\end{verbatim}


The box can in both modes be used without problems many times.


\subsection{Boxes with tagging commands}

We assume in the following that the box contains only well balanced tagging commands and no parts that are \enquote{untagged}. It should be possible to copy the whole box inside a 
\verb+\tagstructbegin+/\verb+\tagstructend+ pair. So the following is fine as box content
\begin{verbatim}
box=\tagstructbegin{...}\tagmcbegin{} balanced content\tagmcend\tagstructend
box=
  \tagmcbegin{}text\tagmcend 
  \tagstructbegin{...}\tagmcbegin{} balanced content\tagmcend\tagstructend
  \tagmcbegin{}text\tagmcend 
\end{verbatim} 
but this not (this case could probably be handled nevertheless with a bit care at least in lua mode)
\begin{verbatim}
box= text\tagmcend\tagstructbegin{...}...\tagstructend\tagmcbegin{}text
\end{verbatim}
and this absolutly unusable:
\begin{verbatim}
box= text\tagmcend\tagstructbegin{...}\tagmcbegin{}text
\end{verbatim}

We also assume that we want to move the structure of the box to the place where the box is used (if the
structure should stay where the box is saved, simply save it and that will happen).
For this we must add a structure that we can stash and label. 

\begin{verbatim}
\tag_mc_end_push: % interrupt an open mc
\tagstructbegin{tag=NonStruct,stash}
 \edef\myboxnum{\tag_get:n{struct_num}} % store structure number
 \setbox\mybox\hbox %or \vbox or ...
   {content}
\tagstructend
\tag_mc_begin_pop:n{}% restart open mc
\end{verbatim}

At the place where the box is then used we also have to inject this structure:

\begin{verbatim}
\tag_mc_end_push: % interrupt an open mc
\tag_struct_use_num:n {\myboxnum} % use structure
\box\mybox             % use box   
\tag_mc_begin_pop:n{}% restart open mc
\end{verbatim}

With pdf\LaTeX{} Boxes with tagging commands can currently be used only once. The tagging commands set labels
and reusing the box gives multiple label warnings.

With lua\LaTeX{} it is possible to reset the attributes as done with the untagged box and then to reuse at least
the content. 

\subsection{Detecting tagging commands}

It is possible to detect if a box contains tagging commands by comparing the state of the mc and structure counter:
\begin{verbatim}
\def\statebeforebox\inteval{\tag_get:n{struct_counter}+\tag_get:n{mc_counter}}
\setbox\mybox ...
%compare numbers against \statebeforebox
\end{verbatim}



\subsection{Putting everything together}

To tag boxes that contain either no tagging commands or only balanced commands the following 
strategy can be used:

\begin{itemize}
\item when storing the box put around it a structure as need by the tagged variant:
\begin{verbatim}
\tag_mc_end_push: % interrupt an open mc
\tagstructbegin{tag=NonStruct,stash} 
\edef\myboxnum{\tag_get:n{struct_num}} % store structure number
 \def\statebeforebox{\inteval{\tag_get:n{struct_counter}+\tag_get:n{mc_counter}}}
 \setbox\mybox\hbox %or \vbox or ...
   {content}
 %check if there is tagging content and store that
\tagstructend
\tag_mc_begin_pop:n{}% restart open mc
\end{verbatim}

\item when using the box the first time
 \begin{itemize}
 \item if it has no tagging commands then reset the attribute and use the box.
  \begin{verbatim}
  The \tagmcresetbox\mybox\box\mybox{} sun
  \end{verbatim}
  The stashed \texttt{NonStruct} structure is then thrown away. 
 \item  if there is a structure then use the stashed structure
  \begin{verbatim}
  \tag_mc_end_push: % interrupt an open mc
  \tag_struct_use_num:n {\myboxnum} % use structure
  \box\mybox             % use box   
  \tag_mc_begin_pop:n{}% restart open mc
  \end{verbatim}
 \end{itemize}
 
 \item if the box is used a second time then throw an error with pdf\LaTeX{}. 
 With lua\LaTeX{} reset the attributes and issue a warning.
 
\end{itemize}

\end{document}
%
%Das \testtagged{\the\mybox}\updateattribute{\the\mybox}\usebox\mybox\ Haus

Die\savebox\mybox{\teststructure{blub}} Ente

\showboxbreadth\maxdimen\showboxdepth\maxdimen\showbox\mybox 


%\testtagged{\the\mybox}
\ExplSyntaxOn
\tag_if_box_tagged:NTF \mybox {yes}{NO}
\ExplSyntaxOff


%
%Das \updateattribute{\the\mybox}\usebox\mybox\ Haus

%Die \emph{\updateattribute{\the\mybox}\usebox\mybox xxx}\ Sonne
\end{document}

\begin{document}
% ## Test 1a
% a simple setbox

\iffalse
Die\setbox0\hbox{gelbe} Ente

Das \box0\ Haus
\fi

% tests (with adobe reader)
% lualatex
% reading:    Die Ente gelbe Das Haus (wrong)
% copy&paste: Die Ente Das gelbe Haus (result ok, but wrong regarding the structure)

% pdflatex 
% reading: Die Ente Das gelbe Haus (ok)
% copy&paste: Die Ente Das gelbe Haus (ok)

% ## Test 1b
% a simple setbox where we interrupt the mc

\iffalse
Die\interruptmc{\setbox0\hbox{gelbe}} Ente

Das \box0\ Haus
\fi

% pdflatex: reading + copy&paste ok: Die Ente Das gelbe Haus (copy&paste miss one space)
% lualatex: reading is missing "gelbe" as it is unmarked and so artifact; copy & paste ok  

% ## Test 1c
% a simple setbox where we interrupt the mc + restart for lualatex

\iffalse
Die\interruptmc{\luamcborder{}\setbox0\hbox{gelbe}\luamcborder{}} Ente

Das \box0\ Haus
\fi

% * pdflatex: reading + copy&paste ok: Die Ente Das gelbe Haus (copy&paste miss one space)
% * lualatex
%   reading + tags:    Die gelbe Ente Das Haus (wrong)
%   copy&paste: Die Ente Das gelbe Haus (result ok, but wrong regarding the structure)

% ## Test 1d
% a simple setbox where we interrupt the mc. With luatex we restart and stash.

\iffalse
Die\interruptmc{\luamcborder{stash,label=mc1d}\setbox0\hbox{gelbe}\luamcborder{}} Ente

Das \mcuse{mc1d}\copy0\ Haus
\fi

% * pdflatex: reading + copy&paste ok: Die Ente Das gelbe Haus (copy&paste miss one space)
% * lualatex: reading + copy&paste Die Ente Das gelbe Haus


% ## Test 1e
% a simple setbox where we interrupt the mc. With luatex we restart and stash.
% we add a dummy structure that we don't use but will be needed if the box has inner structure.

\iffalse
Die\interruptmc{\luamcborder{stash,label=mc1e}\stashstructure{box1e}{\setbox0\hbox{gelbe}\luamcborder{}}} Ente

Das \mcuse{mc1e}\copy0\ Haus
\fi

% * pdflatex + luatex reading + copy& paste ok DieEnte Das gelbe Haus

% ## Test 1f
% same as 1e but we use the box twice

\iftrue
Die\interruptmc{\luamcborder{}\stashstructure{box1e}{\setbox0\hbox{gelbe}\luamcborder{}}} Ente

Das \copy0\ Haus

Das \luamcborder{} \ExplSyntaxOn
\tl_set:Nx  \l__tag_tmpa_tl { \__tag_ref_value:nnn{tagpdf-mc1e}{tagmcabs}{} }
\tl_if_empty:NF\l__tag_tmpa_tl { \__tag_mc_handle_stash:x { \l__tag_tmpa_tl} }
\ExplSyntaxOff\copy0\ \luamcborder{} Haus
\fi



% # Test 2a
% a setbox with a complete, internal structure
% ## handling the mc state
% ### storing
% * with pdflatex we must stop and restart 
% the mc when storing as it will complain 
% about nested mc
% * with lualatex we must also stop and restart 
% the mc when storing as the global
% mc-attribute will confuse following text: the internal 
% \tagmcend will unset the attribute and the text is unmarked.
% ### using
% * with pdflatex the box contains literals. So we must interrupt the mc to avoid to 
% get nested MC in the pdf (no tool seems to detect that ...)
% * lualatex has no problems here.
%
% ## handling the structure
% To move the structure to the place of use we have to stash
% and then use the structure. For the label we add a NonStruct.

\iffalse
Die%
 \interruptmc{\stashstructure{box1}{\setbox0\hbox{\teststructure{gelbe}}}} 
Ente

Das \interruptmc{\tagstructuse{box1}\copy0}\ Haus


% # Test 3
% a setbox with a complete, internal structure which is reused.
%
% reusing a box with structure is not really possible: the structure 
% can not be inserted twice.
% * with pdflatex we get a duplicated mcid: both occurrences are numbered
% as /Span /l3pdf2 BDC in the mcid. It doesn't show up in the tag structure
% in adobe, but is read as  Die Ente Das gelbe gelbe Haus Die Sonne
% * with lualatex the mcid are numbered correctly, but the box is as expected
% inserted into the parent structure and so we have it twice in second paragraph:
% and it is also read as Die Ente Das gelbe gelbe Haus Die Sonne
% (copy & paste again ignore the structure and gives Die Ente Das gelbe Haus Die gelbe Sonne)

Die \interruptmc{\copy0}\ Sonne
\fi

% # Test 4 
% a simple xform 

\iffalse
Die\pdfxformnew{xform1}{}{gelbe} Ente

Das \pdfxformuse{xform1} Haus
\fi

% pdflatex + lualatex are both ok here:
% reading: Die Ente Das gelbe Haus (ok)
% copy&paste: Die Ente Das gelbe Haus (ok)

% test 5 
% xform with mc's 
% this basically doesn't work: pdflatex errors as it can't write the labels 
% and lualatex doesn't find the mc-attributes and so the structure is empty (and there are also no
% BDC marks in the Xobject).
\iffalse
Die%
 \interruptmc{\stashstructure{box2}{\pdfxformnew{xform2}{}{\tagmcbegin{tag=Span}gelbe\tagmcend}}}
Ente

Das \interruptmc{\tagstructuse{box2}{\pdfxformuse{xform2}}}\ Haus
\fi 

% # Test 6
% xform with internal structure
% this basically doesn't work: pdflatex errors as it can't write the labels 
% and lualatex doesn't find the mc-attributes and so the structure is empty.
\iffalse
Die%
 \interruptmc{\stashstructure{box3}{\pdfxformnew{xform3}{}{\teststructure{gelbe}}}}
Ente

Das \interruptmc{\tagstructuse{box3}{\pdfxformuse{xform3}}}\ Haus
\fi 


\end{document}

1
