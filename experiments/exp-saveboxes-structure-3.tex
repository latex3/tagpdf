% !Mode:: "TeX:DE:UTF-8:Main"
\DocumentMetadata{uncompress,testphase=phase-III,xmp=false}
\documentclass{article}
\newsavebox\testbox
\ExplSyntaxOn
\newcommand\teststructure[1]{\tagstructbegin{tag=Span}\tagmcbegin{}#1\tagmcend\tagstructend}
\newcommand\interruptmc[1]{\tag_mc_end_push:#1\tag_mc_begin_pop:n{}}
\newcommand\stashstructure[2]{\tagstructbegin{tag=NonStruct,label=#1,stash}#2\tagstructend}

\sys_if_engine_luatex:TF
 {
   \newcommand\updateattribute[1]{
    \directlua{
     local~type=tex.getattribute(luatexbase.attributes.g__tag_mc_type_attr)
     local~mc=tex.getattribute(luatexbase.attributes.g__tag_mc_cnt_attr)
     ltx.__tag.func.update_mc_attributes(tex.getbox(\int_use:N #1),mc,type)
     }}
 }
 {
   \newcommand\updateattribute[1]{}
 }     
   

\ExplSyntaxOn\makeatletter
% this should go when the format has it ...
\def\newsavebox#1{\@ifdefinable{#1}{\newbox#1 \tl_new:c{l_tag_box_\int_use:N#1_tl}}}

\AddToHookWithArguments{cmd/sbox/before}
  {
   \tl_set:cx {l_tag_box_\int_use:N#1_tl}{\int_eval:n{\tag_get:n{struct_counter}+\tag_get:n{mc_counter}}}
  }
\AddToHookWithArguments {cmd/sbox/after}
  {
    \tl_set:cx {l_tag_box_\int_use:N#1_tl}
      {\int_eval:n{\tag_get:n{struct_counter}+\tag_get:n{mc_counter}-\tl_use:c{l_tag_box_\int_use:N#1_tl}}}    
  }
\makeatother  
\ExplSyntaxOff


\newsavebox\mybox
\begin{document}

\section{Saving and using boxes}

\TeX{} allows to store material in boxes and to use these box once or multiple times in other places.
This poses some challenges to tagging.

\subsection{Boxes without tagging commands}

If no tagging commands were used (or if they were inactive) when the box was stored then
there is no problem to use this box with pdf\LaTeX{}/generic mode in various places. So

\begin{verbatim}
The\savebox\mybox{yellow} duck

The \usebox\mybox{} sun
\end{verbatim}

will produce (assuming para tagging is activated) the paragraph structures \enquote{The duck} and \enquote{The yellow sun}.

With lua\LaTeX{}/lua mode this is different: The nodes in the box will have the mc-attribute value attached which were
active when the box was saved and this value is recorded as kid of the first paragraph. So when the lua code later wanders through the box to find all kids of structure it will also find the content of the \cs{usebox}. This means with lua\LaTeX{} we get the two paragraph structures \enquote{The duck yellow} and \enquote{The sun}. 





%\newbox\myboxb
%xx \setbox\myboxb\hbox{\teststructure{blub}}


%\newbox\myboxb
%x\setbox\myboxb\hbox{blub}

\ExplSyntaxOn
\tag_if_box_tagged:NTF \mybox {YES}{NO}


\savebox\mybox{\teststructure{blub}}

\tag_if_box_tagged:NTF \mybox {YES}{NO}

\ExplSyntaxOff
\end{document}
%
%Das \testtagged{\the\mybox}\updateattribute{\the\mybox}\usebox\mybox\ Haus

Die\savebox\mybox{\teststructure{blub}} Ente

\showboxbreadth\maxdimen\showboxdepth\maxdimen\showbox\mybox 


%\testtagged{\the\mybox}
\ExplSyntaxOn
\tag_if_box_tagged:NTF \mybox {yes}{NO}
\ExplSyntaxOff


%
%Das \updateattribute{\the\mybox}\usebox\mybox\ Haus

%Die \emph{\updateattribute{\the\mybox}\usebox\mybox xxx}\ Sonne
\end{document}

\begin{document}
% ## Test 1a
% a simple setbox

\iffalse
Die\setbox0\hbox{gelbe} Ente

Das \box0\ Haus
\fi

% tests (with adobe reader)
% lualatex
% reading:    Die Ente gelbe Das Haus (wrong)
% copy&paste: Die Ente Das gelbe Haus (result ok, but wrong regarding the structure)

% pdflatex 
% reading: Die Ente Das gelbe Haus (ok)
% copy&paste: Die Ente Das gelbe Haus (ok)

% ## Test 1b
% a simple setbox where we interrupt the mc

\iffalse
Die\interruptmc{\setbox0\hbox{gelbe}} Ente

Das \box0\ Haus
\fi

% pdflatex: reading + copy&paste ok: Die Ente Das gelbe Haus (copy&paste miss one space)
% lualatex: reading is missing "gelbe" as it is unmarked and so artifact; copy & paste ok  

% ## Test 1c
% a simple setbox where we interrupt the mc + restart for lualatex

\iffalse
Die\interruptmc{\luamcborder{}\setbox0\hbox{gelbe}\luamcborder{}} Ente

Das \box0\ Haus
\fi

% * pdflatex: reading + copy&paste ok: Die Ente Das gelbe Haus (copy&paste miss one space)
% * lualatex
%   reading + tags:    Die gelbe Ente Das Haus (wrong)
%   copy&paste: Die Ente Das gelbe Haus (result ok, but wrong regarding the structure)

% ## Test 1d
% a simple setbox where we interrupt the mc. With luatex we restart and stash.

\iffalse
Die\interruptmc{\luamcborder{stash,label=mc1d}\setbox0\hbox{gelbe}\luamcborder{}} Ente

Das \mcuse{mc1d}\copy0\ Haus
\fi

% * pdflatex: reading + copy&paste ok: Die Ente Das gelbe Haus (copy&paste miss one space)
% * lualatex: reading + copy&paste Die Ente Das gelbe Haus


% ## Test 1e
% a simple setbox where we interrupt the mc. With luatex we restart and stash.
% we add a dummy structure that we don't use but will be needed if the box has inner structure.

\iffalse
Die\interruptmc{\luamcborder{stash,label=mc1e}\stashstructure{box1e}{\setbox0\hbox{gelbe}\luamcborder{}}} Ente

Das \mcuse{mc1e}\copy0\ Haus
\fi

% * pdflatex + luatex reading + copy& paste ok DieEnte Das gelbe Haus

% ## Test 1f
% same as 1e but we use the box twice

\iftrue
Die\interruptmc{\luamcborder{}\stashstructure{box1e}{\setbox0\hbox{gelbe}\luamcborder{}}} Ente

Das \copy0\ Haus

Das \luamcborder{} \ExplSyntaxOn
\tl_set:Nx  \l__tag_tmpa_tl { \__tag_ref_value:nnn{tagpdf-mc1e}{tagmcabs}{} }
\tl_if_empty:NF\l__tag_tmpa_tl { \__tag_mc_handle_stash:x { \l__tag_tmpa_tl} }
\ExplSyntaxOff\copy0\ \luamcborder{} Haus
\fi



% # Test 2a
% a setbox with a complete, internal structure
% ## handling the mc state
% ### storing
% * with pdflatex we must stop and restart 
% the mc when storing as it will complain 
% about nested mc
% * with lualatex we must also stop and restart 
% the mc when storing as the global
% mc-attribute will confuse following text: the internal 
% \tagmcend will unset the attribute and the text is unmarked.
% ### using
% * with pdflatex the box contains literals. So we must interrupt the mc to avoid to 
% get nested MC in the pdf (no tool seems to detect that ...)
% * lualatex has no problems here.
%
% ## handling the structure
% To move the structure to the place of use we have to stash
% and then use the structure. For the label we add a NonStruct.

\iffalse
Die%
 \interruptmc{\stashstructure{box1}{\setbox0\hbox{\teststructure{gelbe}}}} 
Ente

Das \interruptmc{\tagstructuse{box1}\copy0}\ Haus


% # Test 3
% a setbox with a complete, internal structure which is reused.
%
% reusing a box with structure is not really possible: the structure 
% can not be inserted twice.
% * with pdflatex we get a duplicated mcid: both occurrences are numbered
% as /Span /l3pdf2 BDC in the mcid. It doesn't show up in the tag structure
% in adobe, but is read as  Die Ente Das gelbe gelbe Haus Die Sonne
% * with lualatex the mcid are numbered correctly, but the box is as expected
% inserted into the parent structure and so we have it twice in second paragraph:
% and it is also read as Die Ente Das gelbe gelbe Haus Die Sonne
% (copy & paste again ignore the structure and gives Die Ente Das gelbe Haus Die gelbe Sonne)

Die \interruptmc{\copy0}\ Sonne
\fi

% # Test 4 
% a simple xform 

\iffalse
Die\pdfxformnew{xform1}{}{gelbe} Ente

Das \pdfxformuse{xform1} Haus
\fi

% pdflatex + lualatex are both ok here:
% reading: Die Ente Das gelbe Haus (ok)
% copy&paste: Die Ente Das gelbe Haus (ok)

% test 5 
% xform with mc's 
% this basically doesn't work: pdflatex errors as it can't write the labels 
% and lualatex doesn't find the mc-attributes and so the structure is empty (and there are also no
% BDC marks in the Xobject).
\iffalse
Die%
 \interruptmc{\stashstructure{box2}{\pdfxformnew{xform2}{}{\tagmcbegin{tag=Span}gelbe\tagmcend}}}
Ente

Das \interruptmc{\tagstructuse{box2}{\pdfxformuse{xform2}}}\ Haus
\fi 

% # Test 6
% xform with internal structure
% this basically doesn't work: pdflatex errors as it can't write the labels 
% and lualatex doesn't find the mc-attributes and so the structure is empty.
\iffalse
Die%
 \interruptmc{\stashstructure{box3}{\pdfxformnew{xform3}{}{\teststructure{gelbe}}}}
Ente

Das \interruptmc{\tagstructuse{box3}{\pdfxformuse{xform3}}}\ Haus
\fi 


\end{document}

1
